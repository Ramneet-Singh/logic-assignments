% Options for packages loaded elsewhere
\PassOptionsToPackage{unicode}{hyperref}
\PassOptionsToPackage{hyphens}{url}
\PassOptionsToPackage{dvipsnames,svgnames,x11names}{xcolor}
%
\documentclass[
  12pt,
  oneside]{article}
\usepackage{amsmath,amssymb}
\usepackage{lmodern}
\usepackage{iftex}
\ifPDFTeX
  \usepackage[T1]{fontenc}
  \usepackage[utf8]{inputenc}
  \usepackage{textcomp} % provide euro and other symbols
\else % if luatex or xetex
  \usepackage{unicode-math}
  \defaultfontfeatures{Scale=MatchLowercase}
  \defaultfontfeatures[\rmfamily]{Ligatures=TeX,Scale=1}
  \setmathfont[]{ccmath}
\fi
% Use upquote if available, for straight quotes in verbatim environments
\IfFileExists{upquote.sty}{\usepackage{upquote}}{}
\IfFileExists{microtype.sty}{% use microtype if available
  \usepackage[]{microtype}
  \UseMicrotypeSet[protrusion]{basicmath} % disable protrusion for tt fonts
}{}
\makeatletter
\@ifundefined{KOMAClassName}{% if non-KOMA class
  \IfFileExists{parskip.sty}{%
    \usepackage{parskip}
  }{% else
    \setlength{\parindent}{0pt}
    \setlength{\parskip}{6pt plus 2pt minus 1pt}}
}{% if KOMA class
  \KOMAoptions{parskip=half}}
\makeatother
\usepackage{xcolor}
\IfFileExists{xurl.sty}{\usepackage{xurl}}{} % add URL line breaks if available
\IfFileExists{bookmark.sty}{\usepackage{bookmark}}{\usepackage{hyperref}}
\hypersetup{
  pdftitle={COL703 - Assignment 4},
  colorlinks=true,
  linkcolor={Maroon},
  filecolor={Maroon},
  citecolor={Blue},
  urlcolor={Blue},
  pdfcreator={LaTeX via pandoc}}
\urlstyle{same} % disable monospaced font for URLs
\usepackage[a4paper,heightrounded]{geometry}
\setlength{\emergencystretch}{3em} % prevent overfull lines
\providecommand{\tightlist}{%
  \setlength{\itemsep}{0pt}\setlength{\parskip}{0pt}}
\setcounter{secnumdepth}{-\maxdimen} % remove section numbering
% Make \paragraph and \subparagraph free-standing
\ifx\paragraph\undefined\else
  \let\oldparagraph\paragraph
  \renewcommand{\paragraph}[1]{\oldparagraph{#1}\mbox{}}
\fi
\ifx\subparagraph\undefined\else
  \let\oldsubparagraph\subparagraph
  \renewcommand{\subparagraph}[1]{\oldsubparagraph{#1}\mbox{}}
\fi
\pagestyle{headings}
\usepackage{amsmath,amsfonts,amsthm,fancyhdr,float,fullpage,mathtools,logicproof}
\floatplacement{figure}{H}
\usepackage{listings}
\lstset{basicstyle=\ttfamily,
  showstringspaces=false,
  commentstyle=\color{red},
  keywordstyle=\color{blue}
}
\newtheorem{theorem}{Theorem}
\newtheorem{corollary}{Corollary}[section]
\newtheorem{lemma}{Lemma}
\theoremstyle{definition}
\newtheorem{remark}{Remark}
\newtheorem{definition}{Definition}
\newcommand\satisfies{\vDash}
\newcommand\entails{\vDash}
\newcommand\derives{\vdash}
\newcommand\arrow{\rightarrow}
\newcommand\numberthis{\addtocounter{equation}{1}\tag{\theequation}}
\ifLuaTeX
  \usepackage{selnolig}  % disable illegal ligatures
\fi

\title{\textbf{COL703 - Assignment 4}}
    \usepackage{authblk}
                                        \author[]{Ramneet Singh}
                                                            \affil{2019CS50445}
                                            \date{November 2022}

\begin{document}
\maketitle

{
\hypersetup{linkcolor=}
\setcounter{tocdepth}{3}
\tableofcontents
}
\newpage

\hypertarget{question-1}{%
\section{Question 1}\label{question-1}}

We have the following three first-order logic formulas representing
points (a), (b) and (c) given in the question: \begin{align*}
    F_1 &:\quad \forall x ( B(x) \arrow \forall y ( \neg S(y,y) \arrow S(x,y) ) ) \\
    F_2 &:\quad \neg \exists x \exists y ( B(x) \land S(x,y) \land S(y,y) ) \\
    F_3 &:\quad \neg (\exists x B(x))
\end{align*}

To show \(F_1 \land F_2 \entails F_3\), we will show that the formula
\(F_1 \land F_2 \land \neg F_3\) is unsatisfiable. First, we Skolemise
each of the conjuncts and convert their matrices into CNF to get the
following three formulas: \begin{align*}
    G_1 &:\quad \forall x \forall y (\quad \neg B(x) \lor S(y,y) \lor S(x,y) \quad) \\
    G_2 &:\quad \forall x \forall y (\quad \neg B(x) \lor \neg S(x,y) \lor \neg S(y,y) \quad) \\
    G_3 &:\quad B(c)
\end{align*} where \(c\) is a fresh constant symbol.

The formula \(F_1 \land F_2 \land \neg F_3\) is equisatisfiable with
\(G_1 \land G_2 \land G_3\). Thus it suffices to give a ground
resolution refutation of \(G_1 \land G_2 \land G_3\). Now we derive
\(\Box\) from ground instances of clauses in the respective matrices of
the formulas \(G_1,G_2,G_3\).

\begin{logicproof}{0}
    \{ \neg B(c), S(c,c) \} & clause of $G_1$'s matrix with $[c/x][c/y]$ \\
    \{ \neg B(c), \neg S(c,c) \} & clause of $G_2$'s matrix with $[c/x][c/y]$ \\
    \{ \neg B(c) \} & 1,2 resolution \\
    \{ B(c) \} & clause of $G_3$'s matrix \\
    \Box & 3,4 resolution
\end{logicproof}

By the ground resolution theorem, \(G_1 \land G_2 \land G_3\) is
unsatisfiable, which implies that \(F_1 \land \land F_2 \land \neg F_3\)
is unsatisfiable. Therefore, we can conclude that
\(F_1 \land F_2 \entails F_3\).

\newpage

\hypertarget{question-2}{%
\section{Question 2}\label{question-2}}

Let \(\sigma\) be a signature and \(\sim\) be a relation on
\(\sigma\)-assignments which satisfies the following properties:

\begin{enumerate}
\def\labelenumi{\arabic{enumi}.}
\tightlist
\item
  If \(\mathcal{A} \sim \mathcal{B}\), then for every atomic formula
  \(F\) we have \(\mathcal{A} \satisfies F\) iff
  \(\mathcal{B} \satisfies F\).
\item
  If \(\mathcal{A} \sim \mathcal{B}\), then for each variable \(x\) we
  have:

  \begin{enumerate}
  \def\labelenumii{\Alph{enumii}.}
  \tightlist
  \item
    For each \(a \in U_{\mathcal{A}}\), there exists
    \(b \in U_{\mathcal{B}}\) such that
    \(A_{[x \;\mapsto\; a]} \sim B_{[x \;\mapsto\; b]}\).
  \item
    For each \(b \in U_{\mathcal{B}}\), there exists
    \(a \in U_{\mathcal{A}}\) such that
    \(\mathcal{A}_{[x \;\mapsto\; a]} \sim \mathcal{B}_{[x \;\mapsto\; b]}\).
  \end{enumerate}
\end{enumerate}

We have the following theorem.

\begin{theorem}
    If $\mathcal{A} \sim \mathcal{B}$, then for every formula $F$ we have $\mathcal{A} \satisfies F$ iff $\mathcal{B} \satisfies F$.
\end{theorem}
\begin{proof}
    Suppose $\mathcal{A} \sim \mathcal{B}$. We will show by induction on the structure of $F$ that for every formula $F$, $\mathcal{A} \satisfies F$ iff $\mathcal{B} \satisfies F$.

    \begin{itemize}
        \item \textbf{Base Case:} $F$ is an atomic formula. From property 1, we have $\mathcal{A} \satisfies F$ iff $\mathcal{B} \satisfies F$.
        \item \textbf{Induction Step:} We consider the following three cases:
        \begin{itemize}
            \item $F = F_1 \lor F_2$. We have:
                \begin{alignat*}{2}
                              &\; \mathcal{A} \satisfies F && \\
                    \text{iff}&\; \mathcal{A} \satisfies F_1 \lor F_2 && \\
                    \text{iff}&\; \mathcal{A} \satisfies F_1 \text{ or } \mathcal{A} \satisfies F_2 &&\; \text{(defn. of } \satisfies \text{)} \\
                    \text{iff}&\; \mathcal{B} \satisfies F_1 \text{ or } \mathcal{B} \satisfies F_2 &&\; \text{(induction hypothesis)} \\
                    \text{iff}&\; \mathcal{B} \satisfies F_1 \lor F_2 &&\; \text{(defn. of } \satisfies \text{)} \\
                    \text{iff}&\; \mathcal{B} \satisfies F &&
                \end{alignat*}
            \item $F = \neg F_1$. We have:
                \begin{alignat*}{2}
                              &\; \mathcal{A} \satisfies F && \\
                    \text{iff}&\; \mathcal{A} \satisfies \neg F_1 && \\
                    \text{iff}&\; \mathcal{A} \not\satisfies F_1 &&\; \text{(defn. of } \satisfies \text{)} \\
                    \text{iff}&\; \mathcal{B} \not\satisfies F_1 &&\; \text{(induction hypothesis)} \\
                    \text{iff}&\; \mathcal{B} \satisfies \neg F_1 &&\; \text{(defn. of } \satisfies \text{)} \\
                    \text{iff}&\; \mathcal{B} \satisfies F
                \end{alignat*}
            \item $F = \exists x F_1$. In this case, we show that $\mathcal{A} \satisfies F$ iff $\mathcal{B} \satisfies F$ by showing both the directions separately.
                \begin{itemize}
                    \item \emph{(only if):} Suppose $\mathcal{A} \satisfies F$. By definition of $\satisfies$, there exists $a \in U_{\mathcal{A}}$ such that $\mathcal{A}_{[x \;\mapsto\; a]} \satisfies F_1$. From Property 2A, we know that there exists $b \in U_{\mathcal{B}}$ such that $\mathcal{A}_{[x \;\mapsto\; a]} \sim \mathcal{B}_{[x \;\mapsto\; b]}$. Given this, we can apply the induction hypothesis on $F_1$ to conclude, from $\mathcal{A}_{[x \;\mapsto\; a]} \satisfies F_1$, that $\mathcal{B}_{[x \;\mapsto\; b]} \satisfies F_1$. From the definition of $\satisfies$, this implies that $\mathcal{B} \satisfies \exists x F_1$. Thus $\mathcal{B} \satisfies F$.
                    \item \emph{(if):} Suppose $\mathcal{B} \satisfies F$. By definition of $\satisfies$, there exists $b \in U_{\mathcal{B}}$ such that $\mathcal{B}_{[x \;\mapsto\; b]} \satisfies F_1$. From Property 2B, we know that there exists $a \in U_{\mathcal{A}}$ such that $\mathcal{A}_{[x \;\mapsto\; a]} \sim \mathcal{B}_{[x \;\mapsto\; b]}$. Given this, we can apply the induction hypothesis on $F_1$ to conclude, from $\mathcal{B}_{[x \;\mapsto\; b]} \satisfies F_1$, that $\mathcal{A}_{[x \;\mapsto\; a]} \satisfies F_1$. From the definition of $\satisfies$, this implies that $\mathcal{A} \satisfies \exists x F_1$. Thus $\mathcal{A} \satisfies F$.
                \end{itemize}
        \end{itemize}
        Since we have shown $\mathcal{A} \satisfies F$ iff $\mathcal{B} \satisfies F$ in all the three cases and they are exhaustive, the induction step is complete.
    \end{itemize}

    From the principle of structural induction, we can conclude that for every formula $F$, $\mathcal{A} \satisfies F$ iff $\mathcal{B} \satisfies F$.
\end{proof}

\newpage

\hypertarget{question-3}{%
\section{Question 3}\label{question-3}}

We have the following four first-order logic formulas representing
points (a), (b), (c) and (d) given in the question: \begin{align*}
    F_1 &:\quad \forall x (\; (\forall y (\; C(x,y) \arrow R(y) \;)) \arrow H(x) \;) \\
    F_2 &:\quad \forall x (\; G(x) \arrow R(x) \;) \\
    F_3 &:\quad \forall x (\; (\exists y (\; G(y) \land C(y,x) \;)) \arrow G(x) \;) \\
    F_4 &:\quad \forall x (\; G(x) \arrow H(x) \;)
\end{align*}

To show \(F_1 \land F_2 \land F_3 \entails F_4\), we will show that
\(F_1 \land F_2 \land F_3 \land \neg F_4\) is unsatisfiable. Skolemising
each of the conjuncts and converting their matrices to CNF, we get the
following formulas. \begin{align*}
    G_1 &:\quad \forall x (\; (C(x,f(x)) \lor H(x)) \;\land\; (\neg R(f(x)) \lor H(x)) \;) \\
    G_2 &:\quad \forall x (\; \neg G(x) \lor R(x) \;) \\
    G_3 &:\quad \forall x \forall y(\; \neg G(y) \lor \neg C(y,x) \lor G(x) \;) \\
    G_4 &:\quad G(c) \;\land\; \neg H(c)
\end{align*} where \(f\) is a fresh function symbol and \(c\) is a fresh
constant symbol.

The formula \(G_1 \land G_2 \land G_3 \land G_4\) is equisatisfiable
with \(F_1 \land F_2 \land F_3 \land \neg F_4\). Thus it suffices to
show a predicate logic resolution refutation of
\(G_1 \land G_2 \land G_3 \land G_4\). The proof is as follows. Note
that we subscript the variables in line \(k\) with \(k\) to ensure we
always resolve clauses with disjoint sets of variables.

\begin{logicproof}
    \{\, \neg G(y_1), \neg C(y_1,x_1), G(x_1) \,\} & clause of $G_3$ \\
    \{\, \neg G(x_2), R(x_2) \,\} & clause of $G_2$ \\
    \{\, \neg G(y_3), \neg C(y_3,x_3), R(x_3) \,\} & 1,2 Res. with $D_1 = \{G(x_1)\}, D_2 = \{\neg G(x_2)\}, \theta=[x_1/x_2][x_3/x_1][y_3/y_1]$
\end{logicproof}

\end{document}
